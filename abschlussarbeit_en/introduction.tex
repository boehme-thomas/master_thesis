%%%%%%%%%%%%%%%%%%%%%%%%%%%%%%%%%%%%%%%%%%%%%%%%%%%%%%%%%%%%%%%%%%%%%%%%%%%%%%%%
% $Id: introduction.tex 111 2015-10-06 12:55:56Z klugeflo $
%%%%%%%%%%%%%%%%%%%%%%%%%%%%%%%%%%%%%%%%%%%%%%%%%%%%%%%%%%%%%%%%%%%%%%%%%%%%%%%%

\chapter{Introduction}
\label{c:introduction}

%%%%%%%%%%%%%%%%%%%%%%%%%%%%%%%%%%%%%%%%%%%%%%%%%%%%%%%%%%%%%%%%%%%%%%%%%%%%%%%%

\infonote{Provide a general introduction to the topic of your thesis here.}
This document serves as a template for your thesis.
At the same time, it contains important information that will help you
to make it easier for you to write your thesis.

\section{Objectives}
\infonote{Describe the objectives of your work here}

\section{Overview}
\infonote{Describe here how you will proceed in the following chapters proceed to achieve the objectives defined above.}
Chapter~\ref{c:content} contains information on how you should
should prepare the content of your thesis.
In Chapter~\ref{c:latex} you can find a short introduction to the packages you may need for this work.
Chapter~\ref{c:conclusion} completes the document.



%%%%%%%%%%%%%%%%%%%%%%%%%%%%%%%%%%%%%%%%%%%%%%%%%%%%%%%%%%%%%%%%%%%%%%%%%%%%%%%%
%%% Local Variables: 
%%% mode: latex
%%% TeX-master: thesis
%%% TeX-PDF-mode: t
%%% End: 
%%%%%%%%%%%%%%%%%%%%%%%%%%%%%%%%%%%%%%%%%%%%%%%%%%%%%%%%%%%%%%%%%%%%%%%%%%%%%%%%
%<!-- Local IspellDict: en -->
